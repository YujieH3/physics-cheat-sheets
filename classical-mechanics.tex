\documentclass[main]{subfiles}

\begin{document}

\begin{multicols}{2}

Stationary action principle
\begin{gather}
    \mathcal{S}(q) = \int \mathcal{L}(q,\dot{q},t) dt
\end{gather}
Equation of motion $q(t)$ can be obtained by $\delta \mathcal{S} = 0$.

Langrange Mechanics

The Euler-Lagrange equation
\begin{gather}
    \boxed{
        \frac{d}{d\tau}\left(\frac{\partial \mathcal{L}}{\partial \dot t}\right) - \frac{\partial \mathcal{L}}{\partial q} = 0
    }
\end{gather}
In many systems $\mathcal{L} = T - V$

Hamiltonian Mechanics
\begin{gather}
    \dv{\mathcal{L}}{t} = 0 \implies \boxed{\mathcal{H}(p,q,t) = p_i\dot{q}^i - \mathcal{L}(q,\dot{q},t)}\\
    \dv{\vb{q}}{t} = \pdv{\mathcal{H}}{\vb{p}}, \quad \dv{\vb{p}}{t} = -\pdv{\mathcal{H}}{\vb{q}}
\end{gather}

Possion brackets
\begin{gather}
    \dv{f}{t} = \poissonbracket{f}{\mathcal{H}} + \pdv{f}{t}\\
    \qq{where}\poissonbracket{f}{g} = \sum_i\qty(\pdv{f}{q_i}\pdv{g}{p_i} - \pdv{g}{q_i}\pdv{f}{p_i})
\end{gather}


\end{multicols}

\textit{Miscellaneous}

Angular frequency $\omega = \dfrac{2\pi}{T} = 2\pi\nu$. (Angular) Wavenumber $k = \dfrac{2\pi}{\lambda} = 2\pi \tilde{\nu}$. $\nu$ aka $f$ is frequency. $\tilde{\nu}$ is wavenumber. In the context of physics wavenumber usually refers to $k$.
$$\frac{\omega}{k} = \frac{\nu}{\tilde{\nu}} = \frac{\lambda}{T}= 1(c)$$

\end{document}
