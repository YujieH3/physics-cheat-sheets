\documentclass[10pt, a4paper]{article}
\usepackage[margin=0.5in]{geometry}
\usepackage{amsfonts, amsmath, amssymb, bm, extarrows}
%\usepackage[none]{hyphenat}%prevent hyphenated words
\usepackage{fancyhdr}
\usepackage{graphicx, float}
\usepackage{multicol}
\usepackage{siunitx}

\allowdisplaybreaks%允许公式换页

\def\D{\partial}
\def\d{\mathrm{d}}
\def\epsilon{\varepsilon}
\def\Phi{\varPhi}

%opening
\title{Electromagnetism Equations}
\author{He Yujie}

\begin{document}

\maketitle

\section{Biot-Savart's Law}

	\begin{align}
		\bm{B} = \dfrac{\mu_0}{4\pi} \oint \dfrac{I \mathrm{d} \bm{l} \times \bm{r}}{r^3}
	\end{align}

\section{Electromagnetic Force}

\begin{multicols}{2}

\subsection{Magnetic force}

	\begin{align}
		\bm{F}_\text{21} = \oint I_\text{2} \mathrm{d} \bm{l}_\text{2} \times \bm{B}_\text{2}
	\end{align}

\subsection{Electromagnetic force}

	\begin{align}
		\bm{F} = q \left(\bm{E} + \bm{v} \times \bm{B}\right)
	\end{align}

\end{multicols}

\section{Dipoles}

\begin{multicols}{2}

\subsection{Electric dipoles}

\begin{align}
	\bm{p} &= 2q\bm{a}\\
	\bm{M} &= \bm{p} \times \bm{E}\\
	U &= - \bm{p} \cdot \bm{E} + C
\end{align}

\subsection{Magnetic dipoles}

\begin{align}
	\bm{\mu} &= I\bm{S}\\ 
	\bm{M} &= \bm{\mu} \times \bm{E}\\
	U &= - \bm{\mu} \cdot \bm{E} + C
\end{align}

\end{multicols}

\section{Capacitance and Inductance}

\begin{multicols}{2}

\subsection{Capacitance}

	\begin{align}
		C &= \dfrac{q}{V} = \dfrac{\epsilon A}{d}\\
		U &= \dfrac{1}{2}CV^2 = \dfrac{1}{2}\dfrac{1}{C}q^2\\
		\left.F_x\right|_q &= -\left(\dfrac{\D U}{\D x}\right)_q\\
		\left.F_x\right|_V &= \left(\dfrac{\D U}{\D x}\right)_V
	\end{align}
	
\subsection{Inductance}
	
	\begin{align}
		LI &= \Phi\\
		L &= -\dfrac{\epsilon}{\left(\dfrac{\d I}{\d t}\right)}\\
		U &= \dfrac{1}{2}LI^2
	\end{align}
	
\end{multicols}

\section{Maxwell Equations}

\begin{multicols}{2}

\subsection{Differential form}

	\begin{align}
		\nabla \cdot \bm{D} &= \rho_\text{free}\\
		\nabla \times \bm{E} &= -\dfrac{\D \bm{B}}{\D t}\\
		\nabla \cdot \bm{B} &= 0\\
		\nabla \times \bm{H} &= \bm{j}_\text{free} + \dfrac{\D \bm{D}}{\D t}
	\end{align}
	
\subsection{Intergral form}

	\begin{align}
		\oint \bm{D} \cdot \mathrm{d} \bm{S} &= q_\text{free}\\
		\oint \bm{E} \cdot \mathrm{d} \bm{l} &= -\dfrac{\D }{\D t}\int \bm{B} \cdot \mathrm{d} \bm{S}\\
		\oint \bm{B} \cdot \mathrm{d} \bm{S} &= 0\\
		\oint \bm{H} \cdot \mathrm{d} \bm{l} &= I_\text{free} + \dfrac{\D }{\D t}\int \bm{D} \cdot \mathrm{d} \bm{S}
	\end{align}
	
\end{multicols}


\section{Energy of Electromagnetic Field}

	\begin{align}
		u_\text{e} &= \dfrac{1}{2} \bm{E} \cdot \bm{D} \left(= \dfrac{1}{2} \epsilon_0 E^2\right)\\
		u_\text{m} &= \dfrac{1}{2} \bm{B} \cdot \bm{H} \left(= \dfrac{1}{2} \dfrac{1}{\mu_0} B^2\right)\\
		\bm{S} &= \bm{E} \times \bm{H}\\
		u &= \dfrac{1}{2}\left(\bm{E} \cdot \bm{D} + \bm{B} \cdot \bm{H}\right)
	\end{align}

\section{*Some Vector Analysis}

	\begin{align}
		\nabla \times (\phi \bm{f}) = \phi(\nabla \times \bm{f}) + (\nabla \phi) \times \bm{f}
	\end{align}


\section{Miscellaneous}

	Electromotive force (emf): 
	$$\epsilon = \oint \bm{E}\cdot\d \bm{l} = -\dfrac{\d \Phi}{\d t}$$
	
	Ohm's law ($\sigma$ being conductivity): 
	$$\bm{J} = \sigma \bm{E},\quad R = \dfrac{1}{\sigma}\dfrac{L}{S}$$
	
	Constitutive relations for electric and magnetic vectors ($\chi$ being susceptibility, $\chi_m$ being magnetic susceptibility): 
	$$\bm{D} = \epsilon_0\bm{E} + \bm{P} = \epsilon_0(1 + \chi)\bm{E} = \epsilon_0\epsilon_r\bm{E}$$ 
	$$\bm{H} = \dfrac{1}{\mu_0}\bm{B} - \bm{M} = \dfrac{1}{\mu_0}\bm{B} - \chi_m \bm{H} = \dfrac{1}{\mu_0(1+\chi_m)}\bm{B} = \dfrac{1}{\mu_0\mu_r}\bm{B}$$

	Speed of light, vacuum dielectric constant and vacuum magnetic permeability:
	$$c^2 = \dfrac{1}{\epsilon_0\mu_0}$$
	
	Actual electromagnetic field:
	$$\text{Re}\left(\bm{E}\right),\quad \text{Re}\left(\bm{B}\right)$$
	\\
	\\
	\begin{center}
		\textbf{Good luck, soldier.}
	\end{center}
\end{document}
