\documentclass[10.5pt, a4paper, oneside]{article}
\usepackage[margin=.7in]{geometry}
\usepackage{amsfonts, amsmath, amssymb, bm, extarrows}
%\usepackage[none]{hyphenat}%prevent hyphenated words
\usepackage{graphicx, float}
\usepackage{multicol}
\usepackage{siunitx}

\allowdisplaybreaks%允许公式换页
\def\D{\partial}

\setcounter{tocdepth}{1}

\numberwithin{equation}{section}%mark: niubi

\begin{document}

%------------------------------------

%------------------------------------


\section{Special Relativity}

\begin{multicols}{2}

\begin{enumerate}
	\item Lorentz transformation
		\begin{align}
		&\left\{\begin{aligned}
		x' &= \gamma (x - u t)\\
		y' &= y\\
		z' &= z\\
		t' &= \gamma (t - \dfrac{u}{c^2}x)
		\end{aligned}\right.
		\\
		&\left\{\begin{aligned}
		x &= \gamma (x' + u t')\\
		y &= y'\\
		z &= z'\\
		t &= \gamma (t' + \dfrac{u}{c^2}x')
		\end{aligned}\right.
		\end{align}
	\item Effects of relativity
		\begin{enumerate}
			\item time dilation
			\begin{align}
				\Delta t = \gamma\Delta t'
			\end{align}
			\item Lorentz contraction
			\begin{align}
				\Delta x = \dfrac{\Delta x'}{\gamma}
			\end{align}
			\item Laue's cylinder
			\begin{align}
				\Delta t = \dfrac{u}{c^2}\Delta x'
			\end{align}
			\item optical Doppler effect\\
			separating (redshift)
			\begin{align}
				\nu = \sqrt{\dfrac{1 - \beta}{1 + \beta}}\nu_0
			\end{align}
			approaching (blueshift)
			\begin{align}
				\nu = \sqrt{\dfrac{1 + \beta}{1 - \beta}}\nu_0
			\end{align}
			\item relativity of simutaneity
			\begin{align}
				\Delta t' &= \gamma (\Delta t - \dfrac{u}{c^2}\Delta x) \nonumber\\
				&= -\gamma\dfrac{u}{c^2}\Delta x
			\end{align}
			\item causality
			\begin{align}
				\Delta t' &= \gamma (\Delta t - \dfrac{u}{c^2}\Delta x) \nonumber\\
				&= \gamma \Delta t (1 - \dfrac{vu}{c^2})\\
				vu &< c^2\nonumber
			\end{align}
			note: where $v$ is signal speed in $S$-frame, $u$ is speed of $S'$ relative to $S$.
		\end{enumerate}
	\item Spacetime diagram
		\begin{align}
			\tan\theta &= \beta < 1\\
			``1'' &= \dfrac{\gamma}{\cos\theta} = \dfrac{1 + \beta^2}{1 - \beta^2}
		\end{align}
	\item Relativistic kinematics
		\begin{align}
			\left\{\begin{aligned}
			v'_x &= \dfrac{v_x - u}{1 - v_x\dfrac{u}{c^2}}\\
			v'_y &= \dfrac{v_y}{\gamma(1 - v_x\dfrac{u}{c^2})}\\
			v'_z &= \dfrac{v_z}{\gamma(1 - v_x\dfrac{u}{c^2})}\\
			\end{aligned}\right.
		\end{align}
	\item Relativistic dynamics
		\begin{enumerate}
			\item mass \& energy
				\begin{align}
					E &= m(v)c^2\\
					E &= K + E_0\\
					m &= \gamma m_0
				\end{align}
			\item momentum
				\begin{align}
					\bm{p} = \gamma m_0 \bm{v}
				\end{align}
			\item mass energy-momentum relation
				\begin{align}
					E &= \sqrt{c^2 p^2 + m_0^2 c^4}
				\end{align}
				for particle $m_0 = 0$\\
				(e.g. photon and neutrino)
				\begin{align}
					E &= cp
				\end{align}
		\end{enumerate}
\end{enumerate}

\end{multicols}

\section{Temperature}
\begin{multicols}{2}
	\begin{enumerate}
		\item Equilibrium State
			\begin{center}
					\begin{tabular}{|c|c|}
					\hline
					system & exchange \\
					\hline
					isolated & $\times$ \\
					\hline
					closed & energy \\
					\hline
					open & mass \& energy \\
					\hline
				\end{tabular}
			\end{center}
		\item Thermodynamics and temperature
		\paragraph{The zeroth law of thermodynamics}
		\textit{Two systems, each in thermal equilibrium with a third (thermometer) are in thermal equilibrium with each other.}
		\item Empirical temperature scales
		\begin{enumerate}
			\item ideal gas scale
				\begin{align}
					T = \SI{273.16}{K}\lim\limits_{p_{tr}\rightarrow 0}\left(\dfrac{p}{p_{tr}}\right)
				\end{align}
			\item Celsius t
				\begin{align}
					T_C = T_{\text{idealgas}} - 273.15
				\end{align}
			\item Fahrenheit
				\begin{align}
					T_F = 32^{\circ} + \dfrac{9}{5}T_C
				\end{align}
		\end{enumerate}
		\item Equation of state
			\begin{align}
				V &\approx V_0(1 + \alpha_V\Delta T - \kappa_T\Delta p)
				\\
				pV &= nRT
			\end{align}
			\begin{enumerate}
				\item coefficient of thermal expansion
				\begin{align}
					\alpha_l &= \dfrac{1}{L}\left(\dfrac{\D L}{\D T}\right)_p
					\\
					\alpha_V &= \dfrac{1}{V}\left(\dfrac{\D V}{\D T}\right)_p = 3\alpha_l
				\end{align}
				\item Isothermal compressibility
				\begin{align}
					\kappa_T = -\dfrac{1}{V}\left(\dfrac{\D V}{\D p}\right)_T
				\end{align}
			\end{enumerate}
	\end{enumerate}
\end{multicols}
%%%%%%%%%%%%%%%%%%%%%%%%%%%%%%%%%%%%%%%%%%%%%%%%
\section{The First Law of Thermodynamics}
\begin{multicols}{2}
	\begin{enumerate}
		\item Internal energy function
			\begin{align}
				\Delta U = U_B - U_A \equiv W_{BA}
			\end{align}
		\item Heat and the first law of thermodynamics
			\begin{enumerate}
				\item heat released by the surrounding system
					\begin{align}
						Q \equiv U_f - U_i -W
					\end{align}
				\item the first law of thermodynamics
					\begin{align}
						\Delta U = W + Q
					\end{align}
			\end{enumerate}
		\item Heat capacity and specific heat
			\begin{enumerate}
				\item heat capacity
				\begin{align}
					C &= \dfrac{1}{m}\lim\limits_{\Delta T \rightarrow 0}\dfrac{\Delta Q}{\Delta T}
					\\
					C_V &= \left(\dfrac{\D U}{\D T}\right)
					\\
					C_p &= 
					\left\{\begin{aligned}
					& \left(\dfrac{\D U}{\D T}\right)_p + p\left(\dfrac{\D V}{\D T}\right)_p
					\\
					& \left.\dfrac{\D(U + pV)}{\D T}\right|_p \equiv \left(\dfrac{\D H}{\D T}\right)_p
					\end{aligned}\right.\\
					C_p &= C_v + nR\\
					C_V &= \dfrac{nR}{\gamma - 1}\\
					C_p &= \dfrac{\gamma nR}{\gamma - 1}
				\end{align}
				\textit{where H is enthalpy.}
				\item ratio of specific heat
				\begin{align}
					\gamma = \dfrac{C_p}{C_v}
				\end{align}
			\end{enumerate}
		\item Adiabatic Equation
		\begin{align}
			pV^\gamma &= C\\
			TV^{\gamma - 1} &= C
		\end{align}
		\item The Carnot Cycle
		\begin{align}
			\eta &= 1 + \dfrac{\Delta Q_{34}}{\Delta Q_{12}} \equiv 1 - \dfrac{Q_2}{Q_1}
			\\
			\eta &= 1 - \dfrac{T_2}{T_1}
		\end{align}
	\end{enumerate}
\end{multicols}
%%%%%%%%%%%%%%%%%%%%%%%%%%%%%%%%%%%%%%%%%%%%%%%%
\section{The Second Law of Thermomdynamics}
\begin{multicols}{2}
	\begin{enumerate}
		\item The Second Law
			\\
			\textbf{Kelvin-Planck statement:}
			No process is possible whose \textbf{sole result} is the absorption from a reservoir and the conversasion of this heat into work.
			\\
			\textbf{Clausius statement:}
			No process is possible whose \textbf{sole result} is the transfer of heat from a cooler to a hotter body.
		\item Carnot Theorem and Thermodynamic Scale
			\begin{align}
				\eta_A &\leqslant \eta_R
				\\
				\tau = \SI{273.16}{K}\dfrac{Q}{Q_\text{tr}} &\equiv \SI{273.16}{K}\dfrac{T}{T_\text{tr}}
			\end{align}
		\item Entropy and Entropy Principle
		\begin{align}
			dS \geqslant \dfrac{\bar{d}Q}{T}
		\end{align}
			\begin{enumerate}
				\item entropy of ideal gas
				\begin{align}
					S = C_V\ln{T} + nR\ln{V} + S_0
				\end{align}
				\item entropy of a reservoir
				\begin{align}
					\Delta S = \dfrac{\Delta Q}{T}
				\end{align}
				\item quasi-static process
				\begin{align}
					\Delta S = \int^{(f)}_{(i)}\dfrac{\bar{d}Q}{T}
				\end{align}
				\item irreversible process\\
					Find reversible process. e.g. isothermal process.
			\end{enumerate}
		\item Thermodynamic potentials*
			\begin{align}
				U &= TS -PV + \mu n\\
				G &= \mu n\\
				dU &= TdS - pdV + \mu dn\\
				dH &= TdS + Vdp + \mu dn\\
				dF &= -SdT - pdV + \mu dn\\
				dG &= -SdT + Vdp + \mu dn
			\end{align}
			Gibbs-Duheim equation
			\begin{align}
				d\mu = -S_m dT + V_m dp
			\end{align}
	\end{enumerate}
\end{multicols}
%%%%%%%%%%%%%%%%%%%%%%%%%%%%%%%%%%%%%%%%
\section{Microscopic Model for Ideal Gas}
\begin{multicols}{2}
	\begin{enumerate}
		\item Three characteristic speed
			\begin{align}
				v_m &= \sqrt{\dfrac{2k_BT}{m}}=1.41\dfrac{k_BT}{m}
				\\
				\bar{v} &= \sqrt{\dfrac{8k_BT}{\pi m}}=1.59\dfrac{k_BT}{m}
				\\
				\bar{v} &= \sqrt{\dfrac{3k_BT}{m}}=1.73\dfrac{k_BT}{m}
			\end{align}
		\item Maxwell velocity distribution
		\item Speed distribution
		\item Equipartition theorem
		\item Effusion
		\item Transport phenomena
	\end{enumerate}
\end{multicols}


\end{document}
