\documentclass{article}

% SUBFILES
\usepackage{subfiles}

% STYLING
\usepackage[margin=.5in]{geometry}
\usepackage{multicol}
\usepackage{titlesec}
\titleformat{\section}
  {\normalfont\scshape}{\thesection}{1em}{}
\usepackage[moderate]{savetrees} % fit in fewer pages

% DRAW DIAGRAMS
\usepackage{tikz}
\usetikzlibrary {arrows.meta,graphs,graphdrawing} 
\usegdlibrary{layered}
% \usetikzlibrary{mindmap}

% TABLES
\usepackage{booktabs}

% MATH
\usepackage{mathtools, amsmath, amssymb, physics}
\usepackage{tensor}
\usepackage{bbold} % for identity one

% CUSTOMIZED COMMANDS
\newcommand{\p}{\partial}
\newcommand{\La}{\mathcal{L}}
\newcommand{\ra}{\rightarrow}
\renewcommand{\dv}[3][]{\frac{d^{#1}{#2}}{d{#3}^{#1}}} % Follow Carroll's derivative notation. \mathrm{d} for one forms

\begin{document}

\tableofcontents

\newpage
\section{Classical Mechanics}
\subfile{classical-mechanics}

\newpage
\section{Field Theory}
\subfile{effective-field-theory}

\newpage
\section{General Relativity}
\subfile{general-relativity}

\newpage
\section{Thermodynamics and Statistical Physics}
\subfile{thermodynamics-and-statistical-physics}

\end{document}